\chapter{Data reconstruction and selection}
\label{sec:Selection}

The analysis of the decays \LbToDpmunuX (signal channel)  and \LbToLcmunu (normalization channel) requires the reconstruction and selection of possible signal candidates.
The term ``signal candidates" implies, that a data sample doesn't contain only the desired signal events after reconstruction and selection, but is also polluted by events from different sources, albeit looking like signal.
These backgrounds can have several reasons: 
One possibility is that the final state particles are randomly combined but fulfil all the applied criteria. 
This kind of background is also known as combinatoric background.
There are furthermore the so-called physical backgrounds.
With this term one summarises physical decays where one either misidentifies a final state particle or only partially reconstruct an event und thus leading to a wrong interpretation of the decay.
As an example for the misidentification consider the decay \decay{\Lb}{\Dz\proton\pim}. If the \pim is now misidentified as a muonthis decay looks exactly like the signal channel of this analysis.
Partially reconstructed events play an important role in the normalisation channel \LbToLcmunu.
There exists also semileptonic \Lb decays into excited \Lcstar states, \decay{\Lb}{\Lcstar\mun\neumb}.
Subsequently, these excited \Lcstar states decay into an \Lc and additional pions or photons.
If one misses these pions and photons the decay looks exactly like \LbToLcmunu.

Since such misidentified decays or combinatoric backgrounds distort the measurementof physical quantities, the event reconstruction and above all the selection aims to reduce these backgrounds as much as possible while keeping as much signal as possible.
At \lhcb, this procedure is done in several steps, described for the present analysis in this chapter, namely the Trigger, the preselction (or stripping) and the offline selection. 
Nonetheless not every background source can be easily eliminated. 
The handling of such issues is part of chapter \ref{sec:Backgrounds}.

\section{Trigger requirements}
Trigger requirements are already applied during data taking to reduce the arising data to a recordable amount.
There exists so called trigger lines for different physics purposes. 
These trigger lines then contain the requirements on the particles' properties.

For the \LbToDpmunuX channel the muon has pass the L0Muon\_TOS line at L0 level. TOS is the abbreviation for Trigger On Signal, i.e. 
