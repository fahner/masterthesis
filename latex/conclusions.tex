\chapter{Conclusion}
\label{sec:Conclusion}
This thesis presents the first branching ratio measurement of the semileptonic decay \LbToDpmunuX in form of a relative branching fraction ratio $\R := \frac{\BR(\LbToDpmunuX)}{\BR(\LbToLcmunu)}$.
The analysed data was collected by the \lhcb experiment during 2011 and 2012 and corresponds to an integrated luminosity of 3\invfb.
The determination of \R requires in principle four quantities: the signal yields of the channels \LbToDpmunuX and \LbToLcmunu as well as the corresponding reconstruction and selection efficiencies.
Since only events with the daughter decays \DToKpi and \LcTopKpi are reconstructed, \R has to be corrected for their respective branching ratios.
These values are taken from other experiments.

Due to the semileptonic nature of signal and normalisation channel, it is not possible to reconstruct the \Lb mass to get the signal yields.
The signal yield of the decay \LbToDpmunuX is determined with a two-dimensional fit to the \Dz\proton mass and the \logIP distribution.
Being a measure of how well the \proton makes a vertex with the \Dz\mun candidate, \logIP enables to distinguish between signal and background from randomly combined protons.
This is very helpful to disentangle the nonresonant \LbToDpmunuX decays from background in the \Dz\proton mass spectrum.
The fit yields in total $(\NDpvalscient \pm \NDperrscient)\cdot 10^{\NDpexpscient}$ signal candidates.
However, the fit does not prevent that other backgrounds than randomly combined protons leak into the signal.
It is shown that the main backgrounds are either fake protons or muons, i.e. decays, that mimic to be \LbToDpmunuX since one of the particles is misidentified as proton or muon.
The leakage of backgrounds into the signal yield is estimated to be $(\NDpBKGratioval \pm \NDpBKGratioerr)\%$.

An anomalous enhancement is observed in the invariant \Dz\proton mass spectrum at a mass of about 2840 \mev.
Different attempts to explain its origin do not provide a final solution.
As it currently seems that this enhancement is anyhow part of an inclusive \LbToDpmunuX decay, its yield is counted as signal.

For the reference channel \LbToLcmunu, the main source of backgrounds are \Lb decays into excited \Lcstar states.
These decays are identified by their lower corrected \Lb mass.
The fit to the corrected mass is based on simulation templates and yields $(\NLcvalscient \pm \NLcerrscient)\cdot 10^{\NLcexpscient}$ \LbToLcmunu events.
It is assumed that all relevant backgrounds are already subtracted before or in the fit.

For the determination of the reconstruction and selection efficiencies, the necessary simulations are reweighted, especially the simulation of the decay \LbToDpmunuX.
The ratio of both efficiencies is determined to be $\effLc/\effDp = \effRatioval \pm \effRatioerr$.

This leads to the final result of the relative branching ratio
\begin{align*}
    \R = \Rval \pm \Rerr\stat \pm \Rsysterr\syst.
\end{align*}
The statistical and the systematic uncertainties are of same order.
The biggest contribution to the systematic uncertainties comes from the reweighting process.
It will certainly become more accurate if a proper theoretical physics description is available.
There is currently no theoretical prediction to be compared with the obtained result.
Since this is the first measurement of \BR(\LbToDpmunuX) anyway, there are no other experimental results for comparison either.

What is left is the question of the origin of the enhancement at low \Dz\proton mass.
There are currently different ongoing analyses at \lhcb observing a similar behaviour in the invariant \Dz\proton mass.
Hopefully, an explanation can be found with combined efforts.
Since this enhancement overlaps with the \LcResI resonance in the \Dz\proton spectrum, the obtained masses and widths of the \LcResI and \LcResII are only preliminary.
These results clearly depend on the parametrisation of the enhancement.
Thus, the obtained properties of the \LcResI and \LcResII are not quoted here again.
To get reliable results here, it has to be clear, what the enhancement's nature is and how large the impact of the \LcRes{(2765)} and \SigmacpRes{(2800)} to the invariant \Dz\proton mass at threshold is.

This thesis complements the results of \babar on the \LcResI and \LcResII resonances \cite{BaBar_D0p}.
With the first measurement of the decay \LbToDpmunuX, it furthermore contributes to the currently largely unexplored \bquark baryon sector.
Both help to understand more of the dynamics of light quarks in the vicinity of the heavy \bquark and might improve effective theories of QCD.
The relative branching ratio \R itself can be included in the background estimations of the \asld or semileptonic $|\Vub|$ measurements presented in Chapter \ref{sec:Theory}, to reduce systematic uncertainties.
This is necessary to hopefully reveal the contributions of New Physics in future, since the current experimental precision is not sufficient for that purpose.
