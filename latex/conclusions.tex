\chapter{Conclusion}
\label{sec:Conclusion}
This thesis presents the first branching ratio measurement of the semileptonic decay \LbToDpmunuX in form of a relative branching fraction ratio $\R := \frac{\BR(\LbToDpmunuX)}{\BR(\LbToLcmunu)}$.
The analysed data has been collected by the \lhcb experiment in the years 2011 and 2012 corresponding to an integrated luminosity of \intlum{3\invfb}.
The determination of \R requires in principle four quantities: the signal yields of the channels \LbToDpmunuX and \LbToLcmunu as well as the corresponding reconstruction and selection efficiencies.
As only events with the daughter decays \DToKpi and \LcTopKpi are reconstructed \R has to be corrected for their respective branching ratios.
These values are taken from other experiments.

Due to the semileptonic nature of signal and normalisation channel, it has not been possible to reconstruct the \Lb mass to get the signal yields.
The signal yield of the decay \LbToDpmunuX has been determined with a two-dimensional fit of the \Dz\proton mass and the \logIP distribution.
Being a measure of how well the \proton makes a vertex with the \Dz\mun candidate, \logIP enables to distinguish between signal and background from randomly combined protons.
This is very helpful to disentangle the nonresonant \LbToDpmunuX decays from background in the \Dz\proton mass spectrum.
The fit yields in total $(\NDpvalscient \pm \NDperrscient)\cdot 10^{\NDpexpscient}$ signal candidates.
However, the fit does not prevent that other backgrounds than randomly combined protons leak into the signal.
It has been shown that the main backgrounds are either fake protons or muons, i.e. decays, that mimic to be \LbToDpmunuX since one of the particles is misidentified as proton or muon.
The leakage of backgrounds into the signal yield has been estimated as $(\NDpBKGratioval \pm \NDpBKGratioerr)\%$.

An anomalous enhancement has been observed in the invariant \Dz\proton mass spectrum at a mass of about 2840 \mev.
Different attempts to explain its origin have been made without a final solution.
As it currently seems that this enhancement is anyhow part of an inclusive \LbToDpmunuX decay, its yield is counted as signal.

For the reference channel \LbToLcmunu, he problem of backgrounds has not been a randomly combined proton but rather that excited \Lcstar states pollute the data.
These decays could be identified by their to lower values shifted corrected \Lb mass.
The fit of the corrected mass is based on simulation templates and yields $(\NLcvalscient \pm \NLcerrscient)\cdot 10^{\NLcexpscient}$.
It is assumed that all relevant backgrounds are already subtracted before or in the fit.

For the determination of the reconstruction and selection efficiencies the necessary simulations had to be reweighted, especially the simulation of the decay \LbToDpmunuX.
The ratio of both efficiencies has been determined to be $\frac{\effLc}{\effDp} = \effRatioval \pm \effRatioerr$.

This leads to the final result of the relative branching ratio
\begin{align*}
    \R = \Rval \pm \Rerr\stat \pm \Rsysterr\syst.
\end{align*}
The statistical uncertainties dominates.
This is not a problem of too few recorded events but rather the \LbToDpmunuX simulation at generator level does not contain enough data.
Thus, the statistiscal error on the efficiencies is quite large, above all the generator level efficiency \effGenDp of the \LbToDpmunuX signal channel amounts to about $20\%$.
The biggest contribution to the systematic uncertainties comes from the reweighting process.
It will certainly become more accurate if a proper theoretical physics description is available.
Unfortunately, the theoretical interest in these kind of decays is rather poor.
There is no prediction, to be compared with the obtained result.
Since this is the first measurement of \BR(\LbToDpmunuX) anyway, there are no other experimental results for comparison either.

What is left is the question of the origin of the enhancement at low \Dz\proton mass.
There are currently different analyses running at \lhcb seeing a similar behaviour in the invariant \Dz\proton mass.
Hopefully, an explanation can be found with combined efforts.
Since this enhancement is overlapping with the \LcResI resonance in the \Dz\proton spectrum, the obtained masses and widths of the \LcResI and \LcResII are only preliminary.
These results are clearly depending on the parametrisation of the enhancement.
Thus, the obtained properties of the \LcResI and \LcResII are not quoted here again.
To get reliable results here, it has to be clear, what the enhancement's nature is and how large the impact of the \LcRes{(2765)} and \SigmacpRes{(2800)} to the invariant \Dz\proton mass at threshold is.
