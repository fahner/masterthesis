\chapter{Conclusion}
\label{sec:Conclusion}
This thesis presents the first branching ratio measurement of the semileptonic decay \LbToDpmunuX in form of a relative branching fraction ratio $\R := \frac{\BR(\LbToDpmunuX)}{\BR(\LbToLcmunu)}$.
The analysed data has been collected by the \lhcb experiment in the years 2011 and 2012 corresponding to an integrated luminosity of \intlum{3\invfb}.
The determination of \R requires in principle four quantities: the signal yields of the channels \LbToDpmunuX and \LbToLcmunu as well as the corresponding reconstruction and selection efficiencies.
As only events with the subdecays \DToKpi and \LcTopKpi are reconstructed \R has to be corrected for their respective branching ratios.
These values are taken from other experiments.

Due to the semileptonic nature of signal and normalisation channel, it has not been possible to reconstruct the \Lb mass to get the signal yields.
The signal yield of the decay \LbToDpmunuX has been determined with a twodimensional fit of the \Dz\proton mass and the \logIP distribution.
Being a measure of how well the \proton makes a vertex with the \Dz\mun candidate, it enables to distinguish between signal and background from randomly combined protons.
This is very helpful to disentangle the nonresonant \LbToDpmunuX decays from background in the \Dz\proton mass spectrum.
The fit yields in total $(\NDpvalscient \pm \NDperrscient)\cdot 10^{\NDpexpscient}$ signal candidates.
However, the fit does not prevent that other backgrounds than randomly combined protons leak into the signal.
It has been shown that the main backgrounds are either fake protons or muons, i.e. decays, that mimic to be \LbToDpmunuX since on of the particles is misidentified as proton or muon.
The leakage of backgrounds into the signal yield has been estimated as $(\NDpBKGratioval \pm \NDpBKGratioerr)\%$.

An anomalous enhancement has been observed in the \Dz\proton mass at a mass of about 2840 \mev.
Different attempts to explain its origin have been made without a final solution.
As it currently seems that this enhancement is anyhow part of an inclusive \LbToDpmunuX decay, its yield is counted as signal.

For the \LbToLcmunu signal yield the problem of backgrounds was not a randomly combined proton but rather that excited \Lcstar states saturate the data, too.
These decays could be identified by their to lower values shifted corrected \Lb mass.
The fit of the corrected mass is based on simulation templates and yields $(\NLcvalscient \pm \NLcerrscient)\cdot 10^{\NLcexpscient}$.
It is assumed that all relevant backgrounds are already subtracted before or in the fit.

For the determination of the reconstruction and selection efficiencies the necessary simulations of the decays included a rather poor physics description, especially of the decay \LbToDpmunuX.
The ratio of both efficiencies has been determined to be $\frac{\effLc}{\effDp} = \effRatioval \pm \effRatioerr$.

This leads to the final result of the relative branching ratio
\begin{align*}
    \R = \Rval \pm \Rerr\stat \pm \Rsysterr\syst.
\end{align*}
Currently the statistical uncertainties dominate.
This not a problem of too few recorded events but rather the \LbToDpmunuX simulation at generator level does not contain enough data\footnote{Mika plans to produce a larger generator level tuple next week. Hopefully, this can be implemented before I hand in this thesis}.
