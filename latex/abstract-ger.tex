In der vorliegenden Arbeit wird zum ersten Mal das Verzweigungsverhältnis des Zerfalls \LbToDpmunuX relativ zum Referenzkanal \LbToLcmunu gemessen.
Die ausgewerteten Daten wurden in den Jahren 2011 und 2012 in Proton-Proton Kollisionen bei einer Schwerpunktsenergie von $7\tev$ beziehungsweise $8\tev$ am \lhcb Experiment gesammelt.
Dies entspricht einer integrierten Luminosität von $3\invfb$.
Mit Hilfe eines zweidimensionalen Fits an die invariante \Dz\proton Masse und an \logIP, einem Maß für die Qualität des \Dz\proton\mun Vertex, werden etwa 23000 \LbToDpmunu Signalereignisse rekonstruiert und extrahiert.
Ein Fit an die korrigierte \Lb Masse ergibt für den Referenzkanal \LbToLcmunu etwa 1,5 Millionen Ereignisse.
Unter Berücksichtigung der Rekonstruktions- und Selektionseffizienzen erhält man damit ein relatives Verzweigungsverhältnis \R von
\begin{align*}
    \R := \frac{\BR(\LbToDpmunuX)}{\BR(\LbToLcmunu)} = \Rval \pm \Rerr\stat \pm \Rsysterr\syst.
\end{align*}
Die Mehrheit der \LbToDpmunuX Zerfälle verläuft dabei nichtresonant, es werden aber auch Zerfälle über die Resonanzen \LcResI und \LcResII gemessen.
Bei Betrachtung der invarianten \Dz\proton Masse wird weiterhin eine unerwartete Häufung von Ereignissen um etwa $2840 \mev$ beobachtet.
Es werden einige mögliche Ursachen hierfür untersucht, dennoch kann keine endgültige Erklärung für diese Häufung präsentiert werden.
Derzeit gibt es keine experimentellen oder theoretischen Vergleichswerte für \R, die hier genannt werden können.

