\chapter{Results}
\label{sec:Results}
This chapter summarises all the ingredients needed for the calculation of the relative branching ratio \R.
As a reminder, the relative branching ratio of the decays \LbToDpmunuX and \LbToLcmunu is calculated by:
\begin{align}
	\R =
	\frac{\BR(\decay{\Lb}{\Dz\proton \mun \neumb X})}{\BR(\decay{\Lb}{\Lc \mun\neumb})}
	 = 
	 \frac{N_{\Dz\proton}}{N_{\Lc}}  
	 \cdot \frac{\epsilon_{\Lc}}{\epsilon_{\Dz\proton}}
	 \cdot \frac{\BR(\decay{\Lc}{p \Km \pip})}{\BR(\decay{\Dz}{\Km\pip})}.
\end{align}

Concerning the signal yield \NDp of the \LbToDpmunuX channel, not all backgrounds could be separated in the signal fit, since it is only able to separate random background.
In section \ref{sec:Backgrounds} additional background contributions like fake protons etc. have been discussed and a background yield \NBgdDp has been assigned.
Thus the final calculation of R is modified to
\begin{align}
	\R =
	 \frac{N_{\Dz\proton}-\NBgdDp}{N_{\Lc}}  
	 \cdot \frac{\epsilon_{\Lc}}{\epsilon_{\Dz\proton}}
	 \cdot \frac{\BR(\decay{\Lc}{p \Km \pip})}{\BR(\decay{\Dz}{\Km\pip})}. \label{eq:R_mod}
\end{align}
Note that for the normalisation fit and the determination of \NLc it is assumed, that all non-negligible backgrounds are considered in the fit.
The efficiency ratio \effRatio has been determined by the use of simulation samples.
These had to be reweighted to better emulate the data distribution.

 
\begin{table}[h]
    \centering
    \caption{Final results needed for the calculation of the relative branching ratio \R according to equation (\ref{eq:R_mod}). The errors correspond to the statistical (first) and systematic (second) precision.}
    \label{tab:table_finalresults}
    $\begin{array}{lr@{\pm}r@{\pm}l}
    \hline
    \text{Variable} & \multicolumn{3}{c}{\text{Value}} \\
    \hline
\multicolumn{4}{l}{\text{\textbf{Signal yields}}} \\
\NDp&(2.29 & 0.09 & 0.07) \cdot 10^{4}\\
\NLc&(1.54 & 0.01 & 0.00) \cdot 10^{6}\\
\multicolumn{4}{l}{\text{\textbf{Backgrounds}}} \\
\NBgdDp&(2.75 & 0.10 & 0.00) \cdot 10^{3}\\
\multicolumn{4}{l}{\text{\textbf{Branching ratios}}} \\
\DBFDp&(3.88 & 0.00 & 0.05) \cdot 10^{-2}\\
\DBFLc&(6.84 & 0.00 & 0.24) \cdot 10^{-2}\\
\multicolumn{4}{l}{\text{\textbf{Efficiencies}}} \\
\effDp&(1.79 & 0.36 & 0.11) \cdot 10^{-3}\\
\effLc&(1.31 & 0.06 & 0.00) \cdot 10^{-3}\\
\frac{\effLc}{\effDp}&(7.30 & 1.50 & 0.50) \cdot 10^{-1}\\

    \hline
    \end{array}$
\end{table}

An overview of all important observables can be found in Table \ref{tab:table_finalresults}.
With these values the relative branching ratio \R gets
\begin{align*}
    \R = \Rval \pm \Rerr\stat \pm \Rsysterr\syst.
\end{align*}
The statistical error dominates compared to the systematic uncertainty.
The main contribution comes from the statistical uncertainty on the efficiency.
Above all the simulation sample at generator level includes only little statistics.
