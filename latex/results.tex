\chapter{Results}
\label{sec:Results}
This chapter summarises all the ingredients needed for the calculation of the relative branching ratio \R.
As a reminder, the relative branching ratio of the decays \LbToDpmunuX and \LbToLcmunu is calculated by:
\begin{align}
	\R =
	\frac{\BR(\decay{\Lb}{\Dz\proton \mun \neumb X})}{\BR(\decay{\Lb}{\Lc \mun\neumb})}
	 = 
	 \frac{N_{\Dz\proton}}{N_{\Lc}}  
	 \cdot \frac{\epsilon_{\Lc}}{\epsilon_{\Dz\proton}}
	 \cdot \frac{\BR(\decay{\Lc}{p \Km \pip})}{\BR(\decay{\Dz}{\Km\pip})}.
\end{align}

Concerning the signal yield \NDp of the \LbToDpmunuX channel, not all backgrounds could be separated in the signal fit, since it is only able to separate random background.
In Chapter \ref{sec:Backgrounds} additional background contributions like fake protons etc. have been discussed and a background yield \NBgdDp has been assigned.
Thus the final calculation of R is modified to
\begin{align}
	\R =
	 \frac{N_{\Dz\proton}-\NBgdDp}{N_{\Lc}}  
	 \cdot \frac{\epsilon_{\Lc}}{\epsilon_{\Dz\proton}}
	 \cdot \frac{\BR(\decay{\Lc}{p \Km \pip})}{\BR(\decay{\Dz}{\Km\pip})}. \label{eq:R_mod}
\end{align}
Note, that for the fit to the reference channel \LbToLcmunu and the determination of \NLc it is assumed, that all non-negligible backgrounds are considered in the fit.
The efficiency ratio \effRatio has been determined by the use of simulation samples.
These had to be reweighted to better emulate the data distribution.

 
\begin{table}[h]
    \centering
    \caption{Final results needed for the calculation of \R according to equation (\ref{eq:R}). The errors correspond to the statistical (first) and systematic (second) precision.}
    \label{tab:table_finalresults.tex}
    $\begin{array}{lr@{\pm}r@{\pm}l}
    \hline
    \text{Variable} & \multicolumn{3}{c}{\text{Value}} \\
    \hline
\multicolumn{4}{l}{\text{\textbf{Signal yields}}} \\
\NDp&(2.29 & 0.02 & 0.07) \cdot 10^{4}\\
\NLc&(1.55 & 0.01 & 0.00) \cdot 10^{6}\\
\multicolumn{4}{l}{\text{\textbf{Backgrounds}}} \\
\NBgdDp&(2.75 & 0.03 & 0.00) \cdot 10^{3}\\
\multicolumn{4}{l}{\text{\textbf{Branching ratios}}} \\
\DBFDp&(3.88 & 0.00 & 0.05) \cdot 10^{-2}\\
\DBFLc&(6.84 & 0.00 & 0.24) \cdot 10^{-2}\\
\multicolumn{4}{l}{\text{\textbf{Efficiencies}}} \\
\effDp&(1.79 & 0.36 & 0.11) \cdot 10^{-3}\\
\effLc&(1.35 & 0.07 & 0.00) \cdot 10^{-3}\\
\frac{\effLc}{\effDp}&(7.50 & 1.60 & 0.50) \cdot 10^{-1}\\

    \hline
    \end{array}$
\end{table}

An overview of all important variables can be found in Table \ref{tab:table_finalresults}.
With these values, me obtains for the relative branching ratio
\begin{align*}
    \R = \Rval \pm \Rerr\stat \pm \Rsysterr\syst.
\end{align*}
The statistical and the systematic uncertainties are of same order.
Concerning the statistical uncertainties, the main contribution comes from the statistical uncertainty on the efficiency ratio.
The systematic uncertainties are dominated by the reweighting of the \LbToDpmunu channel.

As a side effect, the widths and masses of the peaking structures in the invariant \Dz\proton mass spectrum have been measured.
The results are
\begin{align*}
    \LcResI:            && m_{\LcResI}         &= (\LcResImeanval \pm \LcResImeanerr\stat) \mev, \\
                        && \Gamma_{\LcResI}    &= (\LcResIwidthval \pm \LcResIwidtherr\stat) \mev, \\
    \LcResII:           && m_{\LcResII}        &= (\LcResIImeanval \pm \LcResIImeanerr\stat) \mev, \\
                        && \Gamma_{\LcResII}   &= (\LcResIIwidthval \pm \LcResIIwidtherr\stat) \mev, \\
    \text{enhancement}: && m_{\text{enh}}      &= (\Structuremeanval \pm \Structuremeanerr\stat) \mev, \\
                        && \Gamma_{\text{enh}} &= (\Structurewidthval \pm \Structurewidtherr\stat) \mev.
\end{align*}
No studies on systematic uncertainties of these values are done.
As the enhancement and the \LcResI resonance overlap, the widths of the peaking structures obviously depend on the parametrisation of the enhancement and are thus preliminary.
Nonetheless, these (preliminary) results are in agreement with the current PDG values as discussed in Section \ref{sec:Signalyield_D0p}.
