\chapter{Introduction}
\label{sec:Introduction}
Developed about 40 years ago, the Standard Model of Particle Physics very successfully describes the properties of the known elementary particles and their interaction among each other \cite{SM_Glashow, SM_Salam, SM_Weinberg}.
It includes the electromagnetic, the strong and the weak interaction.
Until today it has been probed with tremendous precision.
With the exception of the neutrino masses, no results contradicting its predictions have been found so far.
With the discovery of the Higgs boson in 2012 \cite{Higgs_ATLAS, Higgs_CMS}, confirming the theory of electroweak symmetry breaking, the success of the Standard Model reached its climax.

Nonetheless, there are still open questions, which the Standard Model does not answer:
Where does the asymmetry between matter and antimatter come from\footnote{Admittedly, with the $\mathcal{CP}$-violation the Standard Model provides a framework for matter-antimatter asymmetry, though it is not sufficient to explain the observed asymmetry in the universe.}?
What is the origin of dark matter and of dark energy, making up about 95\% of our universe in total \cite{Planck_Universe}?
How can gravitation be formulated as quantum field theory and be combined with the Standard Model interaction to a more general theory of forces?

The Large Hadron Collider (LHC) of the European Organization for Nuclear Research (CERN) at Geneva, Switzerland is ``the world's largest and most powerful particle accelerator" \cite{CERN_LHC_web} dedicated to tackle those questions.
One of the four big experiments located at the LHC is the \lhcb-experiment.
It is built to precisely measure the Standard Model and to (indirectly) search for physics beyond the Standard Model, often referred to as ``New Physics (NP)", by the study of hadrons containing a \bquark- or \cquark-quark.
These hadrons provide an excellent laboratory for the measurement of New Physics sensitive observables like $\mathcal{CP}$-violation, the CKM matrix elements and many more.
It allows furthermore to do spectroscopy with bound heavy quark states leading to a better understanding of Quantum Chromodynamics (QCD), the theory of the strong interaction responsible for the bindings of quarks to hadrons.
Unfortunately, QCD cannot be treated perturbatively in the energy regime of hadrons.
Thus, theoretical predictions of such \bquark-/\cquark-hadron decays are always forced to make use of approximations for instance of the Heavy Quark Effective Theory (HQET) \cite{HQET_Introduction}.
The quark model of such theories predicts plenty of \bquark-/\cquark-hadron resonances \cite{cBaryons_Predictions}.
Consequently, the more resonances and particles including their properties are measured, the better these effective theories can be adjusted and the better gets the knowledge of the strong interaction.
While the \B factories and hadron colliders made a great progress in studying \bquark mesons, the knowledge of \bquark baryons is rather poor.

This thesis aims to measure the branching fraction of the semileptonic decay \LbToDpmunuX in the form of a relative branching fraction ratio $\R := \frac{\BR(\LbToDpmunuX)}{\BR(\LbToLcmunu)}$ for the first time.
The \Dz\proton subsystem allows a spectroscopical analysis of \Lc resonances, here briefly denoted as \Lcstar via the intermediate reaction \decay{\Lb}{\Lcstar\mun\neumb}, \decay{\Lcstar}{\Dz\proton}.
Furthermore, this decay is a not well understood background in semileptonic searches for New Physics like the $\mathcal{CP}$-violating asymmetry \asld \cite{asld_LHCb} or the measurement of the CKM matrix element $|\Vub|$ in the decay \decay{\Lb}{\proton\mun\neumb} \cite{SL_Vub}.
Hence, with the measurement of \BR(\LbToDpmunuX) it is possible to reduce the uncertainties of those measurements.

This thesis is structured as follows.
Chapter \ref{sec:Theory} introduces the Standard Model of Particle Physics as well as some searches for physics beyond the Standard Model. The last part of this chapter explains experimental techniques and methods used in this analysis.
In Chapter \ref{sec:Detector}, the \lhcb-experiment, where the decays have been recorded, is described.
Afterwards a short overview of the analysis strategy follows in Chapter \ref{sec:Strategy}.
With Chapter \ref{sec:Selection}, the actual analysis starts with the description of the event reconstruction and selection.
It is followed by the determination of the signal yields through fits for the \LbToDpmunuX channel\footnote{In this thesis, the channel \LbToDpmunuX will be also called signal channel and the \LbToLcmunu channel normalisation or reference channel. The corresponding fits are sometimes labeled as signal fit respectively normalisation fit.}
in Chapter \ref{sec:Signalfit} and for the \LbToLcmunu channel in Chapter \ref{sec:Normalisationfit}.
The required reconstruction and selection efficiencies are determined in Chapter \ref{sec:Efficiencies}.
Subsequently, Chapter \ref{sec:Backgrounds} contains a discussion of different backgrounds focusing on the \LbToDpmunuX signal channel.
The systematic uncertainties are studied in Chapter \ref{sec:Systematics} then.
Since an anomalous enhancement appears in the \Dz\proton mass spectrum in the signal fit, Chapter \ref{sec:Structure} tries to find the origin of this enhancement.
In Chapter \ref{sec:Results} all intermediate results are summarised and the relative branching ratio \R is calculated.
Eventually, this thesis ends in Chapter \ref{sec:Conclusion} with the conclusion.

