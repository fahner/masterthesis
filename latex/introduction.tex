\chapter{Introduction}
\label{sec:Introduction}

... some text here ...


This thesis is structured as follows.
Chapter \ref{sec:Theory} introduces the Standard Model of Particle Physics as well as its limitations and the searches challenging it. The last part of this chapter explains experimental techniques and methods used in this analysis.
In Chapter \ref{sec:Detector}, the \lhcb-experiment and above all the detector, where the decays have been recorded, is described.
Afterwards a short overview of the analysis strategy follows in Chapter \ref{sec:Strategy}.
With Chapter \ref{sec:Selection} the actual analysis starts with the description of the event reconstruction and selection.
It is followed by the determination of the signal yields through fits for the \LbToDpmunuX channel 
\footnote{In this thesis, the channel \LbToDpmunuX will be also called signal channel and the \LbToLcmunu channel normalisation channel. The corresponding fits are sometimes labeled as signal fit respectively normalisation fit.}
in chapter \ref{sec:Signalfit} and for the \LbToLcmunu channel in Chapter \ref{sec:Normalisationfit}.
The required reconstruction and selection efficiencies are determined in Chapter \ref{sec:Efficiencies}.
Subsequently, Chapter \ref{sec:Backgrounds} contains a discussion of different backgrounds focusing on the \LbToDpmunuX signal channel.
The systematic uncertainties are studied in Chapter \ref{sec:Systematics} then.
Since an anomalous enhancement appears in the \Dz\proton mass spectrum in the signal fit, Chapter \ref{sec:Structure} tries to lighten the dark a bit and shows some studies on the enhancement.
In Chapter \ref{sec:Results} everything is ready to summarise all intermediate results und finally calculate the relative branching ratio \R.
Eventually, this thesis ends in Chapter \ref{sec:Conclusion} with the conclusion.

