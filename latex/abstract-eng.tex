This thesis presents the first branching ratio measurement of the semileptonic decay \LbToDpmunuX in form of a relative branching fraction ratio $\R := \frac{\BR(\LbToDpmunuX)}{\BR(\LbToLcmunu)}$.
The analysed data has been collected in proton-proton collisions at a centre-of-mass energy of $7 \tev$ respectively $8 \tev$ by the \lhcb experiment in the years 2011 and 2012 corresponding to an integrated luminosity of $3\invfb$.
About 23000 \LbToDpmunu signal events are reconstructed and extracted from a two-dimensional fit to the invariant \Dz\proton mass and \logIP, a measure for the quality of the  \Dz\proton\mun vertex, distribution.
With a fit to the corrected \Lb mass, about 1.5 million \LbToLcmunu events are extracted.
After consideration of reconstruction and selection efficiencies, the relative branching ratio \R is determined as
\begin{align*}
    \R = \frac{\BR(\LbToDpmunuX)}{\BR(\LbToLcmunu)} = \Rval \pm \Rerr\stat \pm \Rsysterr\syst.
\end{align*}
The majority of the \LbToDpmunuX decay width is non-resonant, but there are also contributions from intermediate \LcResI and \LcResII resonances measured.
Furthermore, there is an anomalous enhancement seen in the invariant \Dz\proton mass spectrum.
Its origin is not completely figured out in this thesis, but several cross-checks are presented.
No theoretical predictions or experimental results on \R exists so far for a comparison of the obtained result.
